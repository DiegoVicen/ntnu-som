\documentclass[11pt]{article}
\usepackage[margin=1in]{geometry}  % set the margins to 1in on all sides
\usepackage{graphicx}              % to include figures
\usepackage{amsmath}               % great math stuff
\usepackage{amsfonts}              % for blackboard bold, etc
\usepackage{amsthm}                % better theorem environments
\usepackage{makeidx}               % index
\usepackage[utf8]{inputenc}        % now we have tildes!
\usepackage{graphicx}
\usepackage{wrapfig}
\usepackage[english]{babel}
\usepackage{color}
\usepackage{listings}
\usepackage{indentfirst}

\graphicspath{{img/}}

% Set more space after the equation display to make text more readable
\expandafter\def\expandafter\normalsize\expandafter{%
    \normalsize
    \setlength\abovedisplayskip{100pt}
    \setlength\belowdisplayskip{20pt}
    \setlength\abovedisplayshortskip{3pt}
    \setlength\belowdisplayshortskip{20pt}
}

\definecolor{dkgreen}{rgb}{0,0.6,0}
\definecolor{gray}{rgb}{0.5,0.5,0.5}
\definecolor{mauve}{rgb}{0.58,0,0.82}

\lstset{frame=tb,
  language=Python,
  aboveskip=3mm,
  belowskip=3mm,
  showstringspaces=false,
  columns=flexible,
  basicstyle={\small\ttfamily},
  numbers=left,
  numberstyle=\tiny\color{gray},
  keywordstyle=\color{blue},
  commentstyle=\color{dkgreen},
  stringstyle=\color{mauve},
  breaklines=true,
  breakatwhitespace=true,
  tabsize=3
}

\begin{document}

\begin{center}
  \Huge\textbf{Using Self-Organizing Maps to solve Travelling Salesman
    Problem}\\
  \vspace{1cm}
  \large\textsc{Maximilian Leonard Kleinans \& Diego Vicente Martín}
\end{center}

\section{System Overview}

The goal of this system is to use Self-Organizing Maps to solve the Travelling
Salesman Problem. To achieve it, we will create a population of neurons that
will be iteratively make closer to each of the cities. By defining a 1D
neighbourhood in this 2D space, we will make sure that the population behaves as
an elastic ring that stretches until it occupies all the cities, while it tries
to keep the perimeter as short as possible. Once we achieve it, we only have to
traverse the cities following the ring.\\

The main part of the system is located in \texttt{som.py}, where the main
execution loop is located. In each iteration, we pick a random city among all
the population. With that city, we search for the closest neuron to it. That
neuron, and its neighbourhood, will be updated following the formula:

$$
w_v(s+1) \; \leftarrow \;
w_v(s) + \alpha (s) \cdot \theta (u, v, s) \cdot (D(t) - w_v)
$$

Where $s$ is a iteration in which the neuron $w_u$ has been chosen as the
winner of the city $D(t)$, and the formula is used to update all the neurons
$w_v$ located in its neighbourhood. The parameters in the formula are $\alpha(s)$,
that represents the learning rate; $\theta(u, v, s)$ that represents the
neighbourhood function value of $v$ to respect of $u$ at time $s$. We also have
to remark that thse last parameters decay over time.\\

To do that, we will use different types of decay functions, that are represented
in the \texttt{decay.py} file: static decay (that maintains the same value for
that parameter through all the iterations), linear decay (that linearly
decreases the value of the parameter), or exponential decay (that exponentially
decays the value of the parameter). For convenience, these parameters have been
implemented as classes that can be passed to the computation methods.\\

Apart from these computation files, we also have implemented a file
\texttt{helper.py}, that is used for reading data sets from files and managing
user input, and \texttt{plot.py}, that contains relevant methods from saving
snapshots of the system at different moments of the execution, to see how the
system gets to a certain solution.

\end{document}