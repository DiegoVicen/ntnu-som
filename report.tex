\documentclass[11pt]{article}
\usepackage[margin=1in]{geometry}  % set the margins to 1in on all sides
\usepackage{graphicx}              % to include figures
\usepackage{amsmath}               % great math stuff
\usepackage{amsfonts}              % for blackboard bold, etc
\usepackage{amsthm}                % better theorem environments
\usepackage{makeidx}               % index
\usepackage[utf8]{inputenc}        % now we have tildes!
\usepackage{graphicx}
\usepackage{wrapfig}
\usepackage[english]{babel}
\usepackage{color}
\usepackage{listings}
\usepackage{indentfirst}

\graphicspath{{img/}}

% Set more space after the equation display to make text more readable
\expandafter\def\expandafter\normalsize\expandafter{%
    \normalsize
    \setlength\abovedisplayskip{100pt}
    \setlength\belowdisplayskip{20pt}
    \setlength\abovedisplayshortskip{3pt}
    \setlength\belowdisplayshortskip{20pt}
}

\definecolor{dkgreen}{rgb}{0,0.6,0}
\definecolor{gray}{rgb}{0.5,0.5,0.5}
\definecolor{mauve}{rgb}{0.58,0,0.82}

\lstset{frame=tb,
  language=Python,
  aboveskip=3mm,
  belowskip=3mm,
  showstringspaces=false,
  columns=flexible,
  basicstyle={\small\ttfamily},
  numbers=left,
  numberstyle=\tiny\color{gray},
  keywordstyle=\color{blue},
  commentstyle=\color{dkgreen},
  stringstyle=\color{mauve},
  breaklines=true,
  breakatwhitespace=true,
  tabsize=3
}

\begin{document}

\begin{center}
  \Huge\textbf{Using Self-Organizing Maps to solve Travelling Salesman
    Problem}\\
  \vspace{1cm}
  \large\textsc{Maximilian Leonard Kleinans \& Diego Vicente Martín}
\end{center}

\section{System Overview}

The goal of this system is to use Self-Organizing Maps to solve the Travelling
Salesman Problem. To achieve that, we will create a population of neurons that
is spatially distributed on a ring structure and as the algorithm progresses, neurons are pulled towards the cities. By defining a 1D
neighborhood in this 2D space, we will make sure that the population behaves as
an elastic ring that stretches until it occupies all the cities, while it tries
to keep the perimeter as short as possible. Once we achieve it, we only have to
traverse the cities following the ring to obtain a TSP solution.\\

The main component of the system is located in \texttt{som.py}: First we scale down the data, so that for each city coordinate $ c= (x, y) ~ 0 \le x,y \le 1$ holds true. Further is here the main execution loop located. In each iteration, we pick a random city. Then we compute the best matching unit, that is closest neuron to this city. That
neuron, and its neighborhood, will be updated following the formula:

$$
w_v(s+1) \; \leftarrow \;
w_v(s) + \alpha (s) \cdot \eta (u, v, s) \cdot (D(t) - w_v)
$$

Where $s$ is the iteration in which the neuron $w_u$ has been chosen as the
winner of the city $D(t)$, and the formula is used to update all the neurons
$w_v$ located in its neighborhood. The parameters in the formula are the learning rate $\alpha(s)$,  the neighborhood function  $\eta(u, v, s)$. We also have
to remark that those last parameters decay over time.

To achieve decaying, we will use different types of decay functions, that are represented
in the \texttt{decay.py} file: static decay (that maintains the same value for
that parameter through all the iterations), linear decay (that linearly
decreases the value of the parameter), or exponential decay (that exponentially
decays the value of the parameter). For convenience, these parameters have been
implemented as classes that can be passed to the computation methods.\\

The calculation of a TSP solution from a list of neuron is done as described in chapter 4 in the compendium. That is, for each city we select the closest neuron and then define the city order through the order of the selected neurons. 


Also relevant to the system is to notice that we have implemented two different
neighborhood functions: \texttt{bubble}, constant in the while neighborhood, 0 elsewhere and \text{gaussian}, that creates a Gaussian
distribution around the winner. Both implementations can be found in the file
\texttt{neighborhood.py}.\\

Apart from these computation files, we also have implemented helper methods for reading data sets from files and managing
user input  in \texttt{helper.py}. Furthermore also methods for plotting and saving
snapshots of the system at different moments of the execution, to see how the
system gets to a certain solution.


\section{Tuning the System}

Once the system was implemented, we still had to work with it to get the best
results out of it. Different combinations of parameters lay different results,
and we had to be comprehensive when testing for parameters.\\

As a general conclusion, static decay is generally not a good decay approach to
use. By keeping the parameters with a static value during all the execution is
much harder to make the system converge into a solution. It is still possible to
get decent results, but overall it is not worth to use static decay. If, in the
other hand, we use linear, we can see the system converging much better to a
certain solution: in the beginning the system tries different strategies to then
focus on fixing the small details of the chosen one. More or less the same
approach can be seen with exponential, but thanks to its nature it is able to
stabilize much faster than linear. That's why, overall, the best decay strategy
to use is exponential although linear can also lay great results (specially when
applied to the learning rate alone).\\

In neighborhood functions, the one that lays the best result is the Gaussian
function. Although bubble does not lay bad results, we can see how Gaussian
moves the network in a much softer fashion, due to its gradient to the edges
versus the 1 or 0 values or bubble.\\

Other parameters to be tuned are the initial value of learning rates and
neighbourhood size, although that depends on great part on the data set and the
decay function we are going to use: static will need lower values and we will be
able to use higher values at the beginning if we use a decay strategy. We also
have to think on the size of the population of neurons. As a general rule, we
will need more neurons than cities so we can model the path through all the
cities in the map. As we have been able to test, the optimal size of the
population to get a network with a proper size is about 4-8 times greater than
the city population of the dataset.\\

\end{document}